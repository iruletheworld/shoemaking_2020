\begin{paracol}{2}

    \setlength{\columnsep}{3.5em}
    \setlength{\columnseprule}{0.1pt}
    \colseprulecolor{red}

    {\Large \textbf{WORLD CHAMPIONSHIPS IN SHOEMAKING 2020 --- OFFICIAL CALL FOR COMPETITION}}\endnote{\url{https://shoegazing.com/2019/11/26/world-championships-in-shoemaking-2020-call-for-competition/}}

    \vspace{1em}

    \switchcolumn
    {\Large \textbf{2020年全球制履大赛之正式公开}}\endnote{\url{https://shoegazing.com/2019/11/26/world-championships-in-shoemaking-2020-call-for-competition/}}
    \switchcolumn*

    \begin{figure}[h]
        \centering
        \includegraphics[scale=0.20]{../AldenLongwingLeffot-kopia.jpg}
        \captionsetup{labelformat=empty}
        \caption{\tnr A classic American styled longwing. Perhaps we'll see contest shoes along this style. Picture: Leffot}
        \label{}
    \end{figure}

    \switchcolumn
    \begin{figure}[h]
        \centering
        \includegraphics[scale=0.20]{../AldenLongwingLeffot-kopia.jpg}
        \captionsetup{labelformat=empty}
        \caption{\textit{一对经典的美式长翼。或许这次比赛我们会看到类似的风格。图源:\textnormal{{\tnr Leffot}}}}
        \label{}
    \end{figure}
    \switchcolumn*

    \begin{figure}[h]
        \centering
        \includegraphics[scale=0.25]{../BemerLongwing-kopia.jpg}
        \captionsetup{labelformat=empty}
        \caption{\tnr But longwings can also be made very elegant (although medallion is missing here, if you compare to contest criteria). Picture: Stefano Bemer}
        \label{}
    \end{figure}

    \switchcolumn
    \begin{figure}[h]
        \centering
        \includegraphics[scale=0.25]{../BemerLongwing-kopia.jpg}
        \captionsetup{labelformat=empty}
        \caption{\textit{但是长翼也可以是非常优雅的(尽管相比于比赛要求,这双的鞋头没有雕花)。图源:\textnormal{{\tnr Stefano Bemer}}}}
        \label{}
    \end{figure}

    % \vspace{1em}

    \switchcolumn*

    \bgroup\obeylines

        \textbf{Criteria for shoe:}
        \begin{enumerate}
            \item Longwing brogue derby model (4-6 separate leather pieces, brogueing along all edges of the leather pieces, wingtip and medallion is mandatory, further decorative brogueing and decorations are ok though not necessary, but no contrast stitching)
            \item One left shoe, size UK8 (or corresponding size), maximum 2 width sizes up or down from an acceptable standard width
            \item Reddish brown / burgundy box calf upper (aniline dyed leather)
            \item Leather sole
            \item Hand welted, handmade sole stitch
            \item Dark sole and heel edges, natural coloured bottom (decorations with for example wheels or nails are ok, but no dye or burnish)
            \item Finished inside of the shoe, with sock lining etc.
            \item No branding
            \item Shoes will be displayed without last or shoe tree for the jury, but if shoe tree is provided these will be used when they are showcased during the event
        \end{enumerate}

        % \vspace{1em}
        \newpage

        \switchcolumn
        \textbf{制作要求:}
        \begin{enumerate}
            \item 长翼雕花德比样式。(4至6块单独的皮料。皮料的边缘处的雕花,长翼上的雕花,以及鞋头的雕花皆为必要。可以进行额外的雕花和装饰,但这不为必要。不可以使用异色线。)
            \item 左鞋一只。尺码为英国8码(或者是同等的尺寸)。宽度最多可以为标准尺码的正负两个码。
            \item 红棕色或酒红色小牛皮鞋身(苯胺染皮革)。
            \item 皮质鞋底。
            \item 内外底皆为手缝。
            \item 鞋底边缘和鞋跟边缘皆为暗色。鞋底为自然色(使用滚轮或钉子作装饰是可以的,但不可以染色或磨色)。
            \item 鞋的内部需要用中皮等进行完工。
            \item 不加商标。
            \item 在评判时,鞋子不会装上鞋楦或鞋撑。但如果有提供,则会在公众展示时装上。
        \end{enumerate}
        \switchcolumn*

        Errors in respect to the above specifications will result in deductions of points, 5\% deduction of total points for small errors, 10\% deduction of total points on larger errors. If the shoe does not follow specifications at all, it can be disqualified. Jury decisions on the above cannot be overruled.

        \switchcolumn
        违反以上规则将会导致扣分。小幅度违规将扣除总分的5\%。大幅度违规将扣除总分的10\%。如果鞋子完全不遵照要求,则可能会被取消资格。评委的决定为终审。
        \switchcolumn*

        Competitors can enter both as a company or as a person. All persons that have been part in the making of the shoe should be stated, and which process(es) each person have made.

        \vspace{1em}

        \switchcolumn
        参赛者可以是个人参赛,亦可为公司参赛。所有参与制作的人员都应该被署名,且对应其负责的工序。
        \switchcolumn*

        \textbf{Criteria that will be judged:}
        \begin{itemize}
            \item Degree of difficulty (maximum 10 points per jury member)

            Jury look at how complicated construction methods that have been used, how advanced they have been built both in large and in smaller details, etc.

            \item Execution (maximum 10 points)

            Jury look at how well the various parts of the shoe construction have been made, how neat and clean the work is, how well executed the level of finishing is, etc.

            \item Design / Aesthetics (maximum 5 points)

            Jury look at the overall aesthetics of the shoe, proportions, etc.
        \end{itemize}

        \switchcolumn
        \textbf{评判标准:}
        \begin{itemize}
            \item 制作难度(每位评判最多可判10分)。

            评委将判断制作方法的复杂度,大小细节有多高级,等等。

            \item 做工(最多为10分)。

            评委将判断制作的各方各面,譬如做工有多精细整洁,完工的程度有多高,等等。

            \item 设计/美感(最多为5分)。

            评委将判断鞋子整体的美观度,平衡度,等等。
        \end{itemize}
        \switchcolumn*

        \textbf{Prizes:}
        \begin{itemize}
            \item 1st prize: £3,000. Glass plaquette. Shoe showcased at Isetan Men’s department store in Shinjuku, Tokyo, plus other stores around the world.
            \item 2nd prize: £2,000. Glass plaquette. Shoe showcased at Isetan Men’s in Tokyo, plus other stores around the world.
            \item 3rd prize: £1,000. Glass plaquette. Shoe showcased at Isetan Men’s in Tokyo, plus other stores around the world.
        \end{itemize}

        \vspace{1em}

        \switchcolumn
        \textbf{奖项:}
        \begin{itemize}
            \item 一等奖:3000英镑奖金。玻璃奖座。鞋子将在东京伊势丹新宿店男装部展出。并会在世界上的其它店铺展出。
            \item 二等奖:2000英镑奖金。玻璃奖座。鞋子将在东京伊势丹新宿店男装部展出。并会在世界上的其它店铺展出。
            \item 三等奖:1000英镑奖金。玻璃奖座。鞋子将在东京伊势丹新宿店男装部展出。并会在世界上的其它店铺展出。
        \end{itemize}
        \switchcolumn*

        \textbf{How to enter the competition:}
        Competitors who wish to enter the contest need to register to \href{mailto:shoegazingblog@gmail.com}{shoegazingblog@gmail.com} no later than January 31 2020, send in name/brand under which you wish to enter. Only one entry per competitor. It is free of charge to enter the competition. For any questions, send e-mail to the address above. We encourage brands/makers to take pictures of the making process to be shared after the final on April 25 (but the shoe cannot be shown to the public prior to the event).

        \switchcolumn
        \textbf{如何参赛:}
        参赛者需要在2020年1月31日前(含)电邮
        \href{mailto:shoegazingblog@gmail.com}{shoegazingblog@gmail.com}进行注册。请附上参赛人/参赛品牌的名字。每位参赛者只可以参赛一次。参赛没有费用。若有问题,请电邮上述地址。我们鼓励参赛者拍摄制作过程之照片。这些照片会在4月25日的决赛上分享(但鞋子不可以在此日之前公开)。
        \switchcolumn*

        The competition shoe should be sent to England to be judged and displayed at the London Super Trunk Show 2020 at Saturday April 25. They have to be delivered in London no later than April 22 (we had shoes coming in too late previous years who missed being part of the contest, so please send shoes in time, for example if sent from outside the EU they may stay in customs several days etc).

        \switchcolumn
        参赛鞋应寄送到英格兰以接受评判,及在2020年伦敦超级展会上展出(4月25日,周六)。参赛鞋需要在4月22日前(含)寄送到伦敦。(我们上一年有鞋子来得太迟而无法参赛。故敬请准时送到。若鞋子来自于欧盟以外的地区,它们可能要在海关停留数日,诸如此类。)
        \switchcolumn*

        Address to send the shoe to will be given once competitor send a final confirmation e-mail of finished shoe and is due to ship it.

        \vspace{1em}

        \switchcolumn
        最终寄送地址会在参赛者完成鞋子后,并在发送最终配送确认电邮后告知参赛者。
        \switchcolumn*

        \textbf{Judging process and award ceremony:}
        Jury will review and judge the shoes on Friday april 24, at this stage the shoes will be anonymous\endnote{Jesper Ingevaldsson of Shoegazing will know who enters the contest, due to him taking care of the registration and answering questions. However, he will not know which shoe belongs to whom when reviewing and judging them. For all other parties of the jury, the shoes will be strictly anonymous.}. Note that due to this, competing shoemakers can not show the competition shoes in for example social media until April 25, and they cannot reveal that they are entering the contest. The shoes will be displayed during the London Super Trunk Show event on Saturday April 25, where the award ceremony will take place at 16.00. Then the World Champion in Shoemaking and the podium places will be announced (competitors don't have to be in place themselves), with the full top ten list (the other positions will be revealed later). All competition shoes will also be showcased on Shoegazing and The Shoe Snob’s blogs and many of them in our social media channels.

        \vspace{1em}

        \switchcolumn
        \textbf{评判过程及颁奖仪式:}
        评委们会在4月24日(周五)评判参赛鞋。此时鞋子将匿名评判\endnote{Shoegazing的Jesper Ingevaldsson因为负责注册和答疑的原因,会知道参赛者的资料。但是,他在评判时是不会知道参赛鞋是哪位参赛者的。至于其他评委,参赛鞋将是严格匿名的。}。注意,由于这个原因,参赛者在4月25日前不可以公开参赛鞋,比如说在社交平台上。并且不能公布其是否有参赛。参赛鞋会在4月25日周六在伦敦超级展会上展出。颁奖典礼将在下午4点进行。冠军以及其余前十名将会被公布(参赛者无需亲自到场)。其余名次将随后公布。所有的参赛鞋将会在Shoegazing和Shoe Snob的博客上帖出,并会在其它众多社交平台上展出。
        \switchcolumn*

        \textbf{The jury (preliminary):}
        \begin{itemize}
            \item Philippe Atienza, bespoke shoemaker, third in the World Championships in Shoemaking 2018
            \item Nicholas Templeman, bespoke shoemaker
            \item Yohei Iwasaki\endnote{Translator's note: He is a student of Murata, chief bottom maker of George Cleverly.}, bespoke shoemaker
            \item Sebastian Tarek, bespoke shoemaker
            \item Paolo Scafora, bespoke shoemaker and factory owner
            \item Edmund Schenecker, sponsor, bespoke shoe customer
            \item Kirby Allison, sponsor, founder of The Hanger Project
            \item Gary Tok, sponsor, author of Master Shoemakers
            \item Jesper Ingevaldsson, Shoegazing
            \item Justin FitzPatrick, The Shoe Snob
        \end{itemize}

        \vspace{1em}

        \switchcolumn
        \textbf{评委(暂定):}
        \begin{itemize}
            \item Philippe Atienza,全定制制履师,2018年全球制履大赛季军。
            \item Nicholas Templeman,全定制制履师。
            \item 岩崎阳平\endnote{译注:这位是村田英治的学生,现在在George Cleverly主持底部制作。},全定制制履师。
            \item Sebastian Tarek,全定制制履师。
            \item Paolo Scafora,全定制制履师及工厂所有者。
            \item Edmund Schenecker,赞助人,全定制鞋履顾客。
            \item Kirby Allison,赞助人,The Hanger Project创始人。
            \item Gary Tok,赞助人,《制履大师》的作者。
            \item Jesper Ingevaldsson,来自Shoegazing。
            \item Justin FitzPatrick,The Shoe Snob。
        \end{itemize}
        \switchcolumn*

        The jury decision cannot be overruled.

        \vspace{1em}

        \switchcolumn
        评委的决定为终审。
        \switchcolumn*

        The shoes will be returned to the contestants and can be used as display shoes (for top three, after the tour to Isetan, Japan, and other stores). In the case they need to be shipped back, the contestant need to sort the return shipping with a pre-paid return shipping label.

        \switchcolumn
        参赛鞋将会用作展示鞋,并会返还给参赛者(前三名将在日本伊势丹及其它店铺展览完毕后)。若它们需要被邮递归还,参赛者需要准备预付费的返还邮寄标签。
        \switchcolumn*

    \egroup

\end{paracol}

\theendnotes

