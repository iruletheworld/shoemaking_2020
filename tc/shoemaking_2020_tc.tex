\begin{paracol}{2}

    \setlength{\columnsep}{3.5em}
    \setlength{\columnseprule}{0.1pt}
    \colseprulecolor{red}

    {\Large \textbf{WORLD CHAMPIONSHIPS IN SHOEMAKING 2020 --- OFFICIAL CALL FOR COMPETITION}}\endnote{\url{https://shoegazing.com/2019/11/26/world-championships-in-shoemaking-2020-call-for-competition/}}

    \vspace{1em}

    \switchcolumn
    {\Large \textbf{2020年全球製履大賽之正式公開}}\endnote{\url{https://shoegazing.com/2019/11/26/world-championships-in-shoemaking-2020-call-for-competition/}}
    \switchcolumn*

    \begin{figure}[h]
        \centering
        \includegraphics[scale=0.20]{../AldenLongwingLeffot-kopia.jpg}
        \captionsetup{labelformat=empty}
        \caption{\tnr A classic American styled longwing. Perhaps we'll see contest shoes along this style. Picture: Leffot}
        \label{}
    \end{figure}

    \switchcolumn
    \begin{figure}[h]
        \centering
        \includegraphics[scale=0.20]{../AldenLongwingLeffot-kopia.jpg}
        \captionsetup{labelformat=empty}
        \caption{\textit{一對經典的美式長翼。或許這次比賽我們會看到類似的風格。圖源:\textnormal{{\tnr Leffot}}}}
        \label{}
    \end{figure}
    \switchcolumn*

    \begin{figure}[h]
        \centering
        \includegraphics[scale=0.25]{../BemerLongwing-kopia.jpg}
        \captionsetup{labelformat=empty}
        \caption{\tnr But longwings can also be made very elegant (although medallion is missing here, if you compare to contest criteria). Picture: Stefano Bemer}
        \label{}
    \end{figure}

    \switchcolumn
    \begin{figure}[h]
        \centering
        \includegraphics[scale=0.25]{../BemerLongwing-kopia.jpg}
        \captionsetup{labelformat=empty}
        \caption{\textit{但是長翼也可以是非常優雅的(儘管相比於比賽要求,這雙的鞋頭沒有雕花)。圖源:\textnormal{{\tnr Stefano Bemer}}}}
        \label{}
    \end{figure}

    % \vspace{1em}

    \switchcolumn*

    \bgroup\obeylines

        \textbf{Criteria for shoe:}
        \begin{enumerate}
            \item Longwing brogue derby model (4-6 separate leather pieces, brogueing along all edges of the leather pieces, wingtip and medallion is mandatory, further decorative brogueing and decorations are ok though not necessary, but no contrast stitching)
            \item One left shoe, size UK8 (or corresponding size), maximum 2 width sizes up or down from an acceptable standard width
            \item Reddish brown / burgundy box calf upper (aniline dyed leather)
            \item Leather sole
            \item Hand welted, handmade sole stitch
            \item Dark sole and heel edges, natural coloured bottom (decorations with for example wheels or nails are ok, but no dye or burnish)
            \item Finished inside of the shoe, with sock lining etc.
            \item No branding
            \item Shoes will be displayed without last or shoe tree for the jury, but if shoe tree is provided these will be used when they are showcased during the event
        \end{enumerate}

        % \vspace{1em}
        \newpage

        \switchcolumn
        \textbf{製作要求:}
        \begin{enumerate}
            \item 長翼雕花德比樣式。(4至6塊單獨的皮料。皮料的邊緣處的雕花,長翼上的雕花,以及鞋頭的雕花皆爲必要。可以進行額外的雕花和裝飾,但這不爲必要。不可以使用異色線。)
            \item 左鞋一隻。尺碼爲英國8碼(或者是同等的尺寸)。寬度最多可以爲標準尺碼的正負兩個碼。
            \item 紅棕色或酒紅色小牛皮鞋身(苯胺染皮革)。
            \item 皮質鞋底。
            \item 內外底皆爲手縫。
            \item 鞋底邊緣和鞋跟邊緣皆爲暗色。鞋底爲自然色(使用滾輪或釘子作裝飾是可以的,但不可以染色或磨色)。
            \item 鞋的內部需要用中皮等進行完工。
            \item 不加商標。
            \item 在評判時,鞋子不會裝上鞋楦或鞋撐。但如果有提供,則會在公衆展示時裝上。
        \end{enumerate}
        \switchcolumn*

        Errors in respect to the above specifications will result in deductions of points, 5\% deduction of total points for small errors, 10\% deduction of total points on larger errors. If the shoe does not follow specifications at all, it can be disqualified. Jury decisions on the above cannot be overruled.

        \switchcolumn
        違反以上規則將會導致扣分。小幅度違規將扣除總分的5\%。大幅度違規將扣除總分的10\%。如果鞋子完全不遵照要求,則可能會被取消資格。評委的決定爲終審。
        \switchcolumn*

        Competitors can enter both as a company or as a person. All persons that have been part in the making of the shoe should be stated, and which process(es) each person have made.

        \vspace{1em}

        \switchcolumn
        參賽者可以是個人參賽,亦可爲公司參賽。所有參與製作的人員都應該被署名,且對應其負責的工序。
        \switchcolumn*

        \textbf{Criteria that will be judged:}
        \begin{itemize}
            \item Degree of difficulty (maximum 10 points per jury member)

            Jury look at how complicated construction methods that have been used, how advanced they have been built both in large and in smaller details, etc.

            \item Execution (maximum 10 points)

            Jury look at how well the various parts of the shoe construction have been made, how neat and clean the work is, how well executed the level of finishing is, etc.

            \item Design / Aesthetics (maximum 5 points)

            Jury look at the overall aesthetics of the shoe, proportions, etc.
        \end{itemize}

        \switchcolumn
        \textbf{評判標準:}
        \begin{itemize}
            \item 製作難度(每位評判最多可判10分)。

            評委將判斷製作方法的複雜度,大小細節有多高級,等等。

            \item 做工(最多爲10分)。

            評委將判斷製作的各方各面,譬如做工有多精細整潔,完工的程度有多高,等等。

            \item 設計/美感(最多爲5分)。

            評委將判斷鞋子整體的美觀度,平衡度,等等。
        \end{itemize}
        \switchcolumn*

        \textbf{Prizes:}
        \begin{itemize}
            \item 1st prize: £3,000. Glass plaquette. Shoe showcased at Isetan Men’s department store in Shinjuku, Tokyo, plus other stores around the world.
            \item 2nd prize: £2,000. Glass plaquette. Shoe showcased at Isetan Men’s in Tokyo, plus other stores around the world.
            \item 3rd prize: £1,000. Glass plaquette. Shoe showcased at Isetan Men’s in Tokyo, plus other stores around the world.
        \end{itemize}

        \vspace{1em}

        \switchcolumn
        \textbf{獎項:}
        \begin{itemize}
            \item 一等獎:3000英鎊獎金。玻璃獎座。鞋子將在東京伊勢丹新宿店男裝部展出。並會在世界上的其它店鋪展出。
            \item 二等獎:2000英鎊獎金。玻璃獎座。鞋子將在東京伊勢丹新宿店男裝部展出。並會在世界上的其它店鋪展出。
            \item 三等獎:1000英鎊獎金。玻璃獎座。鞋子將在東京伊勢丹新宿店男裝部展出。並會在世界上的其它店鋪展出。
        \end{itemize}
        \switchcolumn*

        \textbf{How to enter the competition:}
        Competitors who wish to enter the contest need to register to \href{mailto:shoegazingblog@gmail.com}{shoegazingblog@gmail.com} no later than January 31 2020, send in name/brand under which you wish to enter. Only one entry per competitor. It is free of charge to enter the competition. For any questions, send e-mail to the address above. We encourage brands/makers to take pictures of the making process to be shared after the final on April 25 (but the shoe cannot be shown to the public prior to the event).

        \switchcolumn
        \textbf{如何參賽:}
        參賽者需要在2020年1月31日前(含)電郵
        \href{mailto:shoegazingblog@gmail.com}{shoegazingblog@gmail.com}進行註冊。請附上參賽人/參賽品牌的名字。每位參賽者只可以參賽一次。參賽沒有費用。若有問題,請電郵上述地址。我們鼓勵參賽者拍攝製作過程之照片。這些照片會在4月25日的決賽上分享(但鞋子不可以在此日之前公開)。
        \switchcolumn*

        The competition shoe should be sent to England to be judged and displayed at the London Super Trunk Show 2020 at Saturday April 25. They have to be delivered in London no later than April 22 (we had shoes coming in too late previous years who missed being part of the contest, so please send shoes in time, for example if sent from outside the EU they may stay in customs several days etc).

        \switchcolumn
        參賽鞋應寄送到英格蘭以接受評判,及在2020年倫敦超級展會上展出(4月25日,週六)。參賽鞋需要在4月22日前(含)寄送到倫敦。(我們上一年有鞋子來得太遲而無法參賽。故敬請準時送到。若鞋子來自於歐盟以外的地區,它們可能要在海關停留數日,諸如此類。)
        \switchcolumn*

        Address to send the shoe to will be given once competitor send a final confirmation e-mail of finished shoe and is due to ship it.

        \vspace{1em}

        \switchcolumn
        最終寄送地址會在參賽者完成鞋子後,並在發送最終配送確認電郵後告知參賽者。
        \switchcolumn*

        \textbf{Judging process and award ceremony:}
        Jury will review and judge the shoes on Friday april 24, at this stage the shoes will be anonymous\endnote{Jesper Ingevaldsson of Shoegazing will know who enters the contest, due to him taking care of the registration and answering questions. However, he will not know which shoe belongs to whom when reviewing and judging them. For all other parties of the jury, the shoes will be strictly anonymous.}. Note that due to this, competing shoemakers can not show the competition shoes in for example social media until April 25, and they cannot reveal that they are entering the contest. The shoes will be displayed during the London Super Trunk Show event on Saturday April 25, where the award ceremony will take place at 16.00. Then the World Champion in Shoemaking and the podium places will be announced (competitors don't have to be in place themselves), with the full top ten list (the other positions will be revealed later). All competition shoes will also be showcased on Shoegazing and The Shoe Snob’s blogs and many of them in our social media channels.

        \vspace{1em}

        \switchcolumn
        \textbf{評判過程及頒獎儀式:}
        評委們會在4月24日(週五)評判參賽鞋。此時鞋子將匿名評判\endnote{Shoegazing的Jesper Ingevaldsson因爲負責註冊和答疑的原因,會知道參賽者的資料。但是,他在評判時是不會知道參賽鞋是哪位參賽者的。至於其他評委,參賽鞋將是嚴格匿名的。}。注意,由於這個原因,參賽者在4月25日前不可以公開參賽鞋,比如說在社交平臺上。並且不能公佈其是否有參賽。參賽鞋會在4月25日週六在倫敦超級展會上展出。頒獎典禮將在下午4點進行。冠軍以及其餘前十名將會被公佈(參賽者無需親自到場)。其餘名次將隨後公佈。所有的參賽鞋將會在Shoegazing和Shoe Snob的博客上帖出,並會在其它衆多社交平臺上展出。
        \switchcolumn*

        \textbf{The jury (preliminary):}
        \begin{itemize}
            \item Philippe Atienza, bespoke shoemaker, third in the World Championships in Shoemaking 2018
            \item Nicholas Templeman, bespoke shoemaker
            \item Yohei Iwasaki\endnote{Translator's note: He is a student of Murata, chief bottom maker of George Cleverly.}, bespoke shoemaker
            \item Sebastian Tarek, bespoke shoemaker
            \item Paolo Scafora, bespoke shoemaker and factory owner
            \item Edmund Schenecker, sponsor, bespoke shoe customer
            \item Kirby Allison, sponsor, founder of The Hanger Project
            \item Gary Tok, sponsor, author of Master Shoemakers
            \item Jesper Ingevaldsson, Shoegazing
            \item Justin FitzPatrick, The Shoe Snob
        \end{itemize}

        \vspace{1em}

        \switchcolumn
        \textbf{評委(暫定):}
        \begin{itemize}
            \item Philippe Atienza,全定製製履師,2018年全球製履大賽季軍。
            \item Nicholas Templeman,全定製製履師。
            \item 岩崎陽平\endnote{譯註:這位是村田英治的學生,現在在George Cleverly主持底部製作。},全定製製履師。
            \item Sebastian Tarek,全定製製履師。
            \item Paolo Scafora,全定製製履師及工廠所有者。
            \item Edmund Schenecker,贊助人,全定製鞋履顧客。
            \item Kirby Allison,贊助人,The Hanger Project創始人。
            \item Gary Tok,贊助人,《製履大師》的作者。
            \item Jesper Ingevaldsson,來自Shoegazing。
            \item Justin FitzPatrick,The Shoe Snob。
        \end{itemize}
        \switchcolumn*

        The jury decision cannot be overruled.

        \vspace{1em}

        \switchcolumn
        評委的決定爲終審。
        \switchcolumn*

        The shoes will be returned to the contestants and can be used as display shoes (for top three, after the tour to Isetan, Japan, and other stores). In the case they need to be shipped back, the contestant need to sort the return shipping with a pre-paid return shipping label.

        \switchcolumn
        參賽鞋將會用作展示鞋,並會返還給參賽者(前三名將在日本伊勢丹及其它店鋪展覽完畢後)。若它們需要被郵遞歸還,參賽者需要準備預付費的返還郵寄標籤。
        \switchcolumn*

    \egroup

\end{paracol}

\theendnotes

